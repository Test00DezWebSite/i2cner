\section{Introduction} \label{Introduction}
In order to mitigate climate change, and to improve environmental outcomes, many nations are actively seeking to reduce carbon emissions, and have formalized this goal through the Paris Agreements (UNFCCC, 2015). The largest contribution to global greenhouse gas (GHG) emissions, some 73\%, comes from energy consumption, consisting mainly of the transportation, electricity and heat, buildings, manufacturing and construction sectors (WRI, 2020). For developed economy nations rich in natural resources, switching to natural gas or implementing carbon capture and storage (CCS) on fossil fuel plants is one option to reduce GHG emissions, however for developing nations, these options are not always economically feasible, leading to increased emissions (through greater coal use) for rapidly developing nations (IEA, 2019). For Japan, a developed economy nation with no fossil fuel resources, the challenge to reduce GHG emissions is likely to follow a different path, already evidenced by the restart of nuclear reactors and a shift toward large scale renewable energy deployment (IEA, 2019).
Although influenced by the Paris Agreements, Japanese energy policy is governed by the Basic Energy Plan, recently updated to the 5th edition and approved by the Cabinet in July of 2018 (METI, 2018). The Basic Energy Plan outlines national policy toward a new energy system for the years 2030 and 2050, cognizant of limited indigenous resources, the impact of the Fukushima incident and external pressures on energy supplies (ANRE, 2018). The Plan reaffirms the Japanese benchmarks for evaluating the energy system as, first and foremost within the context of energy security, followed by economic efficiency, safety, and a consideration of the environment (summarized as '3E+S'; ibid). Although there are some parallels between the 3E+S goals and the Paris Agreement targets agreed to by Japan, the plan does not set out in detail how the 2050 emission reduction target of 80\% is to be met. This may be due to the lack of evidence for the economically feasible reduction of GHG emissions in each sector, and it has been suggested that electrification of a number of sectors will be required to achieve the ambitious 2050 target, underpinned by electricity generated from low-carbon technologies (Matsuo et al., 2018). For the power sector to achieve such a target, near-zero emissions are required, and early action utilizing existing technologies is preferable to delaying action in preference for future technologies (Ashina et al., 2012). It is likely that a mixture of current and emerging technologies will be employed to achieve carbon reduction targets in Japan. The current candidates include a reinvigoration of the nuclear contribution to final energy demand, deploying CCS to fossil fuel power plants, and the ushering in of the hydrogen economy, underpinned by renewable energy deployment as well as hydrogen imports from abroad (Ashina et al., 2012; Matsuo et al., 2018; METI, 2017). 
The aim of this research is to investigate the likely suite of technologies and their feasibility in meeting Japan's carbon reduction goals, cognizant of energy policy, resource limitations, demand growth, emerging technologies and economic constraints using the TIMES framework. Our dynamic simulations of transition scenarios, by focusing on minimizing the cost of the transition while satisfying CO2 emission constraints, suggest potential economically feasible decarbonization pathways while meeting the increasing near-term electricity demand. Additionally, the significance of key economic parameters of emerging technologies is assessed through sensitivity analysis, in order to highlight the most impactful parameters of each technology and hence guide research and development efforts focused on these technologies.
