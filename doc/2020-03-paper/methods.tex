\section{Methodology} \label{method}
% what is TIMES, what are we doing, common features of all scenarios(obj, constraints, time frame, technology details, how are results presented/assessed
\subsection{TIMES Model Description}
The \gls{TIMES} model generator is designed to model dynamic energy systems and simulate transition scenarios as a mixed-integer programming linear optimization problem that is subject to a primary objective function and additional constraints. The generation, trade, refinement, storage, and supply of energy commodities across multiple sectors and multiple regions is modeled using a wide variety of in-built commodity and process types. Emissions can be associated with energy commodities or processes as emission coefficient per unit commodity produced or consumed. 

The objective function in our simulations is the overall cost of the transition. The major constrains in our simulations are the demand for electricity (see table \ref{demand}), emission constraints based on Japan's \gls{INDC} (see table \ref{co2-limits}), and feasible nameplate capacity deployment limits (see table \ref{caplim}). Hence each simulation is focused on minimizing the transition cost while attempting to meet the increasing electricity demand and achieving the required emission cuts using, a combination of technologies of various readiness levels. 

While electricity demand in the near term is expected to grow, long term electricity demand is expected to plateau, or even decrease, due to Japan's aging population. However, precisely quantifying this rate of decrease is challenging as there is potential for increased electrification of  transportation and industrial sectors. Hence, post-2030, we have assumed a demand curve based on the likelihood of increased electrification resulting in increasing demand for electricity, but the demand eventually plateaus due to the aging population. The unique initial condition of the post-Fukushima Japanese electricity supply system is captured using \gls{EDMC} data from 2013-2016. Long term impacts of factors such as the retirement of the existing nuclear fleet and the new deployment of emerging, potentially disruptive technology is assessed by simulating the system until 2100. The carbon cost of each technology is accounted for using an emission coefficient that incorporates both direct emissions and life cycle emissions (averaged over the entire lifetime output) for every technology, as applicable.


\begin{table}[!ht]
	\caption{Demand increase over time.}
	\vspace{0.1in}
	\begin{tabularx}{\textwidth}{p{0.5\textwidth} p{0.5\textwidth}}
		\hline
\textbf{Year} & \textbf{Annual demand increase} \\
\hline
2017-2030 & 1.7 \% \\
2031-2050 & 1.0 \% \\
2051-2070 & 0.5 \% \\
2070-2100 & 0.0 \% \\
\hline 
	\end{tabularx}
\label{demand}
\end{table}

\begin{table}[!ht]
	\caption{CO$_2$ constraints.}
	\vspace{0.1in}
	\begin{tabularx}{\textwidth}{p{0.1\textwidth} p{0.22\textwidth}p{0.16\textwidth} p{0.4\textwidth}}
		\hline
\textbf{Year} & \textbf{Emission limit} & \textbf{Base Year} & \textbf{Reduction from base year} \\
\hline
2030 & 438 Mt CO$_2$-eq. & 2013 & 26 \% \\
2050 & 75 Mt CO$_2$-eq. & 1990 & 80 \% \\
2100 & 75 Mt CO$_2$-eq. & 1990 & 80 \% \\
\hline 
	\end{tabularx}
\label{co2-limits}
\end{table}

To explore possible pathways to curbing \gls{GHG} emissions, we simulated different transition scenarios with different combinations of technologies enabled for deployment. The first set of technologies includes conventional technologies such as  \gls{USC},\gls{lng}, solar photovoltaic, wind energy (with onshore, offshore-fixed and offshore-floating considered separately) and utility-scale lithium-ion battery storage. New deployments of oil-fuelled power plants are disabled due to the declining use of oil for electricity generation in Japan, as it is contrary to Japan's goal of energy security and independence under the Basic Energy Plan. The second set of technologies considered includes emerging carbon-neutral technologies, namely \gls{CCS} and utility-scale hydrogen power. Based on their technological readiness and scalability, \gls{CCS}-based coal and natural gas,\gls{AEC},\gls{PEMEC},\gls{PEMFC}, and \gls{SOFC} were selected for simulation. It is also important to consider the impact of nuclear energy as it has extremely low life-cycle emissions and a high capacity factor,thereby being more reliable than renewables. However, nuclear power faces extremely low public acceptance in Japan after the Fukushima Daiichi accident, and its future in Japan is highly uncertain. Hence, scenarios where new nuclear reactor deployment is enabled and disabled must be juxtaposed to assess the potential role of nuclear in decarbonization of the electricity generation sector. Finally, the long-term impact of nascent technologies on the hydrogen economy is considered in an additional scenario. In this scenario, the potential of the commercialization of \gls{SOEC} and \gls{PWS} post-2050 is explored. Thus, five scenarios are simulated to explore possible transition pathways under the aforementioned conditions, and these are detailed in table \ref{scen-table}.

ADD TECH PARAMS,ASSUMPTIONS, FIX TABLE.
The economic data, emission coefficients, nameplate capacity limits and growth rates incorporated in the model are detailed in tables \ref{eco}, \ref{caplim}, and \ref{growrate} respectively. 


%approach, goal, what are we hoping to learn

\begin{table}[!ht]
	\caption{Scenario definition.}
	\vspace{0.1in}
	\begin{tabularx}{\textwidth}{p{0.15\textwidth} p{0.25\textwidth} p{0.25\textwidth} p{0.35\textwidth}}
\hline 
\textbf{Scenario}& \textbf{Emerging tech.} & \textbf{New nuclear} & \textbf{Nascent tech.}\\
                 & \textbf{enabled} & \textbf{enabled} & \textbf{enabled}\\
                  \hline
%1               &   \xmark       &      \greencheck     \\ 
%2               & \xmark       &         \xmark       \\ 
%3               &   \greencheck     &      \greencheck     \\ 
%4               &   \greencheck     &         \xmark       \\
1               &  No       &         No     &     No  \\ 
2               &   No       &      Yes     &     No  \\ 
3               &   Yes     &         No      &     No   \\
4               &   Yes     &      Yes     &     No  \\ 
5               &   Yes     &      Yes     &     Yes  \\ 
\hline
	\end{tabularx}
\label{scen-table}
\end{table}



\subsection{Sensitivity analysis}
%approach, goal, what are we hoping to learn

