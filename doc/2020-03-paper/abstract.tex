\begin{abstract}

We simulated five possible pathways to meeting 2030 and 2050 emission targets within the Japanese electricity supply sector using a single-region \gls{TIMES} model. Key features of our simulations include the incorporation of novel technologies like hydrogen electrolyzers, carbon caputre, photochemical water splitting, and emerging solar technologies, long-term impact assessment up to 2100, and inclusion of life-cycle emission and learning curves for novel technologies. Results indicate that a hybrid approach using hydrogen from renewable energy-based electrolysis and nuclear power, is cost-effective and provides long-term emission reduction along with energy security. Hydrogen from renewables, nuclear power, wind, and novel solar technologies emerge as key emission reduction technologies, while natural gas with carbon capture plays a minor role.

\end{abstract}
