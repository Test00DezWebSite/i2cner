\documentclass[review]{elsarticle}

\usepackage{lineno}
\usepackage{xspace}
\modulolinenumbers[5]

\journal{Annals of Nuclear Energy}

%% `Elsevier LaTeX' style
\bibliographystyle{elsarticle-num}
%%%%%%%%%%%%%%%%%%%%%%%

%%%% packages and definitions (optional)
\usepackage{placeins}
\usepackage{booktabs} % nice rules (thick lines) for tables
\usepackage{microtype} % improves typography for PDF
\usepackage{hhline}
\usepackage{amsmath}

%\usepackage[demo]{graphicx}
%\usepackage{caption}
%\usepackage{subcaption}

\usepackage{booktabs}
\usepackage{threeparttable, tablefootnote}

\usepackage{tabularx}
\newcolumntype{b}{>{\hsize=1.0\hsize}X}
\newcolumntype{s}{>{\hsize=.5\hsize}X}
\newcolumntype{m}{>{\hsize=.75\hsize}X}
\newcolumntype{x}{>{\hsize=.25\hsize}X}

\graphicspath{ {figures/} }

% tikz %
\usepackage{tikz}
\usetikzlibrary{positioning, arrows, decorations, shapes}

\usetikzlibrary{shapes.geometric,arrows}
\tikzstyle{process} = [rectangle, rounded corners, minimum width=3cm, minimum height=1cm,text centered, draw=black, fill=blue!30]
\tikzstyle{object} = [ellipse, rounded corners, minimum width=3cm, minimum height=1cm,text centered, draw=black, fill=green!30]
\tikzstyle{arrow} = [thick,->,>=stealth]

% hyperref %
\usepackage[hidelinks]{hyperref}

\begin{document}
\begin{frontmatter}
\title{Strategies for thorium fuel cycle transition in the SD-TMSR}
\end{frontmatter}
\section*{Highlights}
\begin{itemize}
	\item Five initial fissile loadings: HALEU, Pu mixed with HALEU, reactor-grade Pu, TRU, and $^{233}$U have been studied for transitioning to the thorium fuel cycle in the SD-TMSR.
	\item The MSR burnup routine provided by SERPENT-2 is used to simulate the online reprocessing and refueling in the SD-TMSR.
	\item The dynamics of $k_{eff}$, major isotopes mass, and essential safety parameters have been investigated.
	\item The neutron energy spectrum shift during the reactor operation was calculated.
	\item Temperature coefficient of reactivity evolution during SD-TMSR operation are investigated.
\end{itemize}
\end{document}

