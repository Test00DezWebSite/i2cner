
%
% Copyright 2007, 2008, 2009 Elsevier Ltd
%
% This file is part of the 'Elsarticle Bundle'.
% ---------------------------------------------
%
% It may be distributed under the conditions of the LaTeX Project Public
% License, either version 1.2 of this license or (at your option) any
% later version.  The latest version of this license is in
%    http://www.latex-project.org/lppl.txt
% and version 1.2 or later is part of all distributions of LaTeX
% version 1999/12/01 or later.
%
% The list of all files belonging to the 'Elsarticle Bundle' is
% given in the file `manifest.txt'.
%

% Template article for Elsevier's document class `elsarticle'
% with numbered style bibliographic references
% SP 2008/03/01
%
%
%
% $Id: elsarticle-template-num.tex 4 2009-10-24 08:22:58Z rishi $
%
%
%\documentclass[preprint,12pt]{elsarticle}
\documentclass[answers,11pt]{exam}

% \documentclass[preprint,review,12pt]{elsarticle}

% Use the options 1p,twocolumn; 3p; 3p,twocolumn; 5p; or 5p,twocolumn
% for a journal layout:
% \documentclass[final,1p,times]{elsarticle}
% \documentclass[final,1p,times,twocolumn]{elsarticle}
% \documentclass[final,3p,times]{elsarticle}
% \documentclass[final,3p,times,twocolumn]{elsarticle}
% \documentclass[final,5p,times]{elsarticle}
% \documentclass[final,5p,times,twocolumn]{elsarticle}

% if you use PostScript figures in your article
% use the graphics package for simple commands
% \usepackage{graphics}
% or use the graphicx package for more complicated commands
\usepackage{graphicx}
% or use the epsfig package if you prefer to use the old commands
% \usepackage{epsfig}

% The amssymb package provides various useful mathematical symbols
\usepackage{amssymb}
% The amsthm package provides extended theorem environments
% \usepackage{amsthm}
\usepackage{amsmath}

% The lineno packages adds line numbers. Start line numbering with
% \begin{linenumbers}, end it with \end{linenumbers}. Or switch it on
% for the whole article with \linenumbers after \end{frontmatter}.
\usepackage{lineno}

% I like to be in control
\usepackage{placeins}

% natbib.sty is loaded by default. However, natbib options can be
% provided with \biboptions{...} command. Following options are
% valid:

%   round  -  round parentheses are used (default)
%   square -  square brackets are used   [option]
%   curly  -  curly braces are used      {option}
%   angle  -  angle brackets are used    <option>
%   semicolon  -  multiple citations separated by semi-colon
%   colon  - same as semicolon, an earlier confusion
%   comma  -  separated by comma
%   numbers-  selects numerical citations
%   super  -  numerical citations as superscripts
%   sort   -  sorts multiple citations according to order in ref. list
%   sort&compress   -  like sort, but also compresses numerical citations
%   compress - compresses without sorting
%
% \biboptions{comma,round}

% \biboptions{}


% Katy Huff addtions
\usepackage{xspace}
\usepackage{color}

\usepackage{multirow}
\usepackage[hyphens]{url}


\usepackage[acronym,toc]{glossaries}
\newacronym{<++>}{<++>}{<++>}
\newacronym{I$^2$CNER}{I$^2$CNER}{International Institute for Carbon Neutral Energy Research}
\newacronym{MARKAL}{MARKAL}{MARKet ALlocation}
\newacronym{TIMES}{TIMES}{The Integrated MARKAL-EFOM System}
%\newacronym[longplural={metric tons of heavy metal}]{MTHM}{MTHM}{metric ton of heavy metal}



\makeglossaries

%\journal{Annals of Nuclear Energy}

\begin{document}

%\begin{frontmatter}

% Title, authors and addresses

% use the tnoteref command within \title for footnotes;
% use the tnotetext command for the associated footnote;
% use the fnref command within \author or \address for footnotes;
% use the fntext command for the associated footnote;
% use the corref command within \author for corresponding author footnotes;
% use the cortext command for the associated footnote;
% use the ead command for the email address,
% and the form \ead[url] for the home page:
%
% \title{Title\tnoteref{label1}}
% \tnotetext[label1]{}
% \author{Name\corref{cor1}\fnref{label2}}
% \ead{email address}
% \ead[url]{home page}
% \fntext[label2]{}
% \cortext[cor1]{}
% \address{Address\fnref{label3}}
% \fntext[label3]{}

\title{The role of current and emerging technologies in meeting Japan's mid- to long-term carbon reduction goals\\
\large Response to Review Comments}
\author{Anshuman Chaube, Andrew Chapman, Akari Minami, James Stubbins, Kathryn Huff}

% use optional labels to link authors explicitly to addresses:
% \author[label1,label2]{<author name>}
% \address[label1]{<address>}
% \address[label2]{<address>}


%\author[uiuc]{Kathryn Huff}
%        \ead{kdhuff@illinois.edu}
%  \address[uiuc]{Department of Nuclear, Plasma, and Radiological Engineering,
%        118 Talbot Laboratory, MC 234, Universicy of Illinois at
%        Urbana-Champaign, Urbana, IL 61801}
%
% \end{frontmatter}
\maketitle
\section*{Review General Response}
We would like to thank the reviewers for their detailed assessment of
this paper. Your suggestions, clarifications, and comments have resulted in 
changes which improved the paper.


\begin{questions}
        \section*{Reviewer 1}

        %---------------------------------------------------------------------
        
         \question "The largest contribution to global greenhouse gas (GHG) emissions, some 73\%, comes from energy consumption in the transportation, electricity and heat, buildings, manufacturing, and construction sectors [2]."
Since the focus of the paper is on electricity, the share of GHG should be presented separately.
        
        \begin{solution}
                Thank you for these comments. We appreciate your detailed review. For the comment above, the following text has been added to the paper
        \end{solution}

    %---------------------------------------------------------------------
        
         \question "Japan has been rapidly deploying renewables and has attempted to transition away from nuclear power since the Fukushima Daiichi accident [3]. Yet Japan is also considering reinvigorating its nuclear power sector, evidenced by the restart of its nuclear reactors [4]."
 away from nuclear energy is not true and contradicts itself directly in the following sentence
        
        \begin{solution}
                Thank you for these comments. For the comment above, the following text has been added to the paper to make the transition between the two statements less jarring. However, the narrative was fundamentally true for a while, as indicated by the /slow-restart of nuclear power plants and increasing reliance on coal, as shown by TEPCO(CHECK PUBLISHER NAME)'s Annual Electricity Review.
                
        \end{solution}
        
                      %---------------------------------------------------------------------

         \question "lessons learnt from these decarbonisation simulations could inform global energy policy by delineating potential transition pathways, identifying key novel technologies, and their most significant parameters."
	added value not clearly elaborated in comparison to existing literature        
        \begin{solution}
                Thank you for these comments. We appreciate your detailed review. For the comment above, the following text has been added to the paper
        \end{solution}	
        
    %---------------------------------------------------------------------
        
         \question "We account for the carbon cost of each technology using an emission coefficient that incorporates both direct emissions and life cycle emissions averaged over the entire operating lifetime for each technology"	
         Describe in more detail how calculations are made here. How do the factors enter the model? Statically? Separately for the production, use and disposal phases? Where does the data come from? CO2 only or CO2 equivalents? 
        
        \begin{solution}
                Thank you for these comments.  For the comment above, the following text has been added to the paper
        \end{solution}
        
                    %---------------------------------------------------------------------

        
         \question "or hydrogen electrolysers and fuel cells, life-cycle emissions from just the stack are considered, as balance-of-plant emissions from utility scale hydrogen depend strongly on the type of plant and the source of energy used for electrolysis."
Does this mean that H2 goes in emission-free? Then this would have to be clearly stated.
        \begin{solution}
                Thank you for these comments. We appreciate your detailed review. For the comment above, the following text has been added to the paper
        \end{solution}      
        
   %---------------------------------------------------------------------
          
         \question Figure 1-5 starts at 2013, until when is historical data shown and when does simulation start?
        \begin{solution}
                Thank you for these comments. We appreciate your detailed review. For the comment above, the following text has been added to the paper
        \end{solution}      
              
        
        \section*{Reviewer 2}

        %---------------------------------------------------------------------
        \question The Paris agreement converged on Intended Nationally Determined Contributions on carbon footprint reduction. What are the parameters a country such as Japan consider while determining such targets?
        
        \begin{solution}
                Thank you for these comments. We appreciate your detailed review, which has certainly improved the paper. Regarding the parameters that must be considered, we have referenced the Paris Agreements which state the international considerations, and the Japanese Basic Energy Plan and the 3E+S approach, which we have summarised briefly in the introduction chapter. Since the focus of our paper is not on energy policy review and analysis, further details of these parameters are left in the references for the sake of brevity.
        \end{solution}

        %---------------------------------------------------------------------

        \question I have some apprehensions regarding the long-term horizon of 2100 for investigating the impact of the technologies. Such a long duration will involve many uncertainties. Further, the technology growth has been exponential in the last century. Therefore, it seems plausible that new technologies will be introduced in the energy mix. How the current paper deals with such uncertainties and technology disruptions?
        
        \begin{solution}
                 Thank you for your comment. We have conducted a thorough literature review to identify key technologies that could play a significant role in decarbonisation at scale. Likely changes in key technology parameters such as investment cost, emission coefficient etc. are incorporated dynamically through learning curves obtained from our references, and using TIMES interpolation features. Many of these data sources identify median values and ranges of key parameters in the original references. Beyond that, it is true that novel technologies may emerge in the future, and they will be incorporated in any future simulations that we conduct. The final paragraph of the Conclusion section has been revised to indicate the same :
                 
                 
        \end{solution}
        
                \question Carbon capture and sequestration is not a new technology. It has been utilized since long. Please highlight the chronological evolution of CCS and Hydrogen technologies to put your contributions in context. In fact, a table showing such chronological evolution for all the suggested technologies viz. nuclear power, wind, solar must be provided. 
        
        \begin{solution}
                 Thank you for your comment. While carbon capture technology has been available in some form for a variety of applications since early 2000s, high-efficiency utility-scale carbon capture and sequestration plants have not been developed beyond the pilot stage in any nation, including Japan. This is further indicated by our data references provided by the EIA, which projects the first commercial prototype to become available in 2022. We have amended the initial reference to "carbon capture and sequestration" to "utility-scale carbon capture and sequestration" for further clarity.
                 
                 
        \end{solution}
        
        %--------------------------------------------------------------------
      
                        \question It has been argued that the life cycle cost associated with wind and photovoltaic power may far outweigh  their benefits. For example, what about the disposal of the used wind turbine blades and their environmental impact? How have you considered such impacts in the time horizon? Also, provide a tabular summary of the expected environmental impact of the suggested technologies. 
        
        \begin{solution}
        
                 Thank you for your comment. One form of environmental impact, namely the direct and lifecycle CO$_2$ emissions, has been incorporated for all methods of electricity generation and storage using TIMES's emission coefficient feature. For all energy sources including renewables like wind and solar, the references for lifecycle emissions include the carbon cost of materials, manufacturing, and disposable as and when applicable. Since the focus of our paper is on modelling energy systems while focussing on \textbf{carbon emissions}, other forms of environmental impact such as recycling, acidification, human toxicity etc. are beyond the scope of our paper.
                 
                 
        \end{solution}

        %--------------------------------------------------------------------
                        \question In a country such as Japan, which has been badly impacted by nuclear disasters, how can nuclear power be part of the suggested energy mix for carbon neutral future? Line no. 18-19 in P-2 directly contradicts line no 34-36 in the abstract page.
        
        \begin{solution}
        
                 Thank you for your comment. We have touched upon this issue in our discussion, by identifying a need for improving safety standards within the Japanese regulatory framework, and for the need of engendering public acceptance by creating trust in nuclear power. Since the focus of our paper is not on policy and creating public acceptance, further discussion on this issue is beyond the scope of our paper.
                 
                 The lines 18-19 in page 2 have been revised as follows for further clarity:
                 
                 
        \end{solution}

        %--------------------------------------------------------------------

  
        %--------------------------------------------------------------------
                        \question  Should not the 3E+S benchmark be 3E+SS where the second S stands for social acceptance? I believe social acceptance is the first requirement of any energy systems expected for countries governed by dictatorial methods.
        
        \begin{solution}
        
                 Thank you for your comment. We have simply referenced the policy as named by the Japanese government. We are not responsible for the nomenclature and are unable to modify its name for the sake of precision. We have referenced the need for social acceptance in the Discussion section, especially when discussing nuclear power.
                 
                 
        \end{solution}      
        
        
               %--------------------------------------------------------------------
                        \question   Please provide enumerated lists for research gaps, contributions and novelty of the paper at the end of section 2 under a subsection 2.1 Research gaps and contributions.
                                
        \begin{solution}
        
                 Thank you for your comment. We have identified research gaps by highlighting the lack of clarity in the transition pathway, the emergence of new technologies, the incorporation of carbon emission coefficients for all technologies, and the use of sensitivity analysis in determining key parameters across our abstract, introduction, and conclusion. A separate section has not been added for brevity.
                 
                 
        \end{solution}     
        
              %--------------------------------------------------------------------
                        \question   Start section 4.2 provide a table detailing a summary of parameters and range of their variations for which the sensitivity analysis is performed.
                                
        \begin{solution}
        
                 Thank you for your comment. The same table is present in the draft at the end of Section 4.2. We have not placed the table at the beginning of the section for the sake of readability, specifically so that the reader may first be introduced to the concept and scope of our sensitivity analysis and the parameters therein, before diving into the detailed ranges of variation.
                 
                 
        \end{solution} 
                      %--------------------------------------------------------------------
                        \question   The Acknowledgement section mentions the funding source as the ministry of culture and others. What about the impact of suggested technologies on the culture and tradition of the country. Sustainable energy goals must meet social, economic, environmental and technological sustainability. Culture is an integral part of social sustainability of any energy resource. 
                                
        \begin{solution}
        
                 Thank you for your comment. We have simply referenced a source of funding. The focus of the paper is on modelling the electricity supply sector of a developed nation using TIMES while reducing carbon emissions. Cultural impacts are beyond the scope of our paper.
                 
                 
        \end{solution} 
                              %--------------------------------------------------------------------
                        \question   Finally, I appreciate the authors for writing such a  thought provoking paper. I personally believe Nuclear Power is the only energy source which can provide sustainable energy for a long-time horizon like 2100. Starting from small energy sources such as nuclear battery to GW size generators, nuclear power has all in it. It would be great if the authors investigate the major impediments to a nuclear dominated energy scenario. Can the political and social constructs of the nation and for that matter the whole world accept nuclear power? Since Reliability theory says that all systems eventually fail, should not we give nuclear power a fair chance?  
                                
        \begin{solution}
        
                 Thank you for your comment. We appreciate the depth of your question. We have referenced nuclear-specific challenges throughout the paper in discussing the associated cost, the safety concerns especially after Fukushima-Daiichi, and the lack of public acceptance. However, the focus of the paper is on modelling the electricity supply sector using TIMES while reducing carbon emissions. Investigating impediments to any specific technology and proposing solutions to such impediments is beyond the scope of our paper, which aims to conduct an technologically-impartial analysis into decarbonisation transition scenarios.
                 
                 
        \end{solution} 
        
                                      %--------------------------------------------------------------------

        \section{Reviewer 3}
        
         \question Table 1. Please, remove the refrence from the table and include it in the table caption.
        \begin{solution}
                Thank you for these comments. We appreciate your detailed review. The requested changes have been made to Table 1, and the caption now reads
                
        \end{solution}  
                                      %--------------------------------------------------------------------

         \question Line 14 of page 15, please remove bold text.
        \begin{solution}
                Thank you for these comments. The requested text has been switched from bold to normal. To make similar text congruent with this change, all other transition costs have been changed from bold to normal.
                
        \end{solution}  
                                      %--------------------------------------------------------------------
 
\end{questions}
%\bibliographystyle{elsarticle-num}
%\bibliography{../2019-Scenarios}
\end{document}

%
% End of file `elsarticle-template-num.tex'.
