\section{Background and literature review} \label{litreview}
The Paris Agreement commits individual nations to significant carbon reduction over time through the \gls{INDC} mechanism \cite{united_nations_framework_convention_on_climate_change_unfccc_submission_2015}. Japan, as a signatory to the Paris Agreements, has submitted an INDC with the following goals and timelines: reduce GHG emissions by 26\% compared to 2013 levels by 2030, and reduce overall GHG emissions by 80\% or more by 2050, through the ``development and diffusion of low-carbon technologies and transition to a low-carbon socio-economic structure" \cite{united_nations_framework_convention_on_climate_change_unfccc_submission_2015}. 
Aware of these targets, many researchers have evaluated Japan's future energy system using a variety of modelling approaches, which we review below. Using the \gls{MARKAL} model, considering the uncertainties of technology development, Ozawa et al. found that hydrogen will play a major role in the future energy system, reliant on both nuclear power and \gls{CCS} to reduce electricity sector emissions to nearly zero by 2050 \cite{ozawa_hydrogen_2018}. Recognising the benefits that renewable energy will play in reducing carbon emissions, and the issues of intermittency of renewables, Li et al. explored the role of hydrogen as a storage medium through power-to-gas approaches in Kyushu, Japan. Their study identified that power-to-gas can increase the effective utilisation rate of renewable energy, and the use of hydrogen in the gas network, effectively pairing the electricity and gas networks, overcomes current renewable electricity curtailment issues \cite{li_potential_2019}. Cognizant of the Japanese government's strategic approach to carbon reduction out to 2050 via energy system reform, Chapman and Pambudi also identify a strong role for nuclear, renewables, and hydrogen under a carbon constrained, optimal cost MARKAL/TIMES simulation approach \cite{chapman_strategic_2018}. Considering Japan's economic conditions and demographic trends, such as moderate GDP growth and rapid population ageing, Kuriyama et al. suggest that 2030 targets can be met or exceeded (with up to 42\% GHG reduction) with limited renewable energy growth and a 15\% contribution from nuclear, or without nuclear, under a renewable growth scenario \cite{kuriyama_can_2019}. However, these trends and energy system changes will likely be insufficient to meet the more ambitious 2050 targets. 

Taking a more holistic view in line with the Japanese government's 3E+S targets, ambitious research and development to enable high levels of renewable deployment is necessary to not only meet deep emission reduction goals, but to also reduce Japanese dependence on imported fuels, which would affect both CCS and nuclear deployment rates into the future. Consensus on policy options and priorities also has a large influence on modelling outcomes for the Low Carbon Navigator, which assesses Japanese energy and emission options out to 2050 \cite{moinuddin_japan_2019}. A seminal work by Sugiyama et al. harmonises a number of modelling approaches for Japan's long-term (up to 2050) climate change mitigation options, utilising national and global general and partial equilibrium models \cite{sugiyama_japans_2019}. Model results are contrasted under six scenarios which incorporate a baseline, a range of emissions reductions (50-80\%), and regional obligations for global models (ibid.) under the Paris Agreement target of an 80\% reduction. Each of the models assessed recognise the importance of renewable energy deployment by 2050, notably hydro, solar, and wind, with varying contributions from nuclear energy and fossil fuels, predominantly natural gas. Additionally, for Japan, the option of importing carbon-free hydrogen was identified as potentially playing a critical role \cite{akimoto_estimates_2010, matsuo_global_2013, oshiro_diffusion_2015, sugiyama_japans_2019}. Many studies consider hydrogen a critical part of Japan's low-carbon energy transition, as it can improve energy security, it can be produced from multiple sources, and lacks emissions from fuel combustion \cite{iida_hydrogen_2019}. Global modelling efforts consider the incorporation of long-distance international transport of hydrogen with end uses dominated by passenger and freight fuel cell vehicles and power generation, via both mixed and direct combustion. Electricity from hydrogen is estimated to emerge in Japan from 2030 onwards, as nuclear and coal-fired power generation reduce towards 2050 \cite{ishimoto_significance_2017}. From a policy standpoint, Japan has committed to achieving a hydrogen society, with the primary goal of cost parity of hydrogen with competing fuels, which requires a three-fold reduction in cost by 2030 and further reductions into the future \cite{nagashima_japans_2018}. Under the Basic Hydrogen Strategy, the Japanese government aims to realise low cost hydrogen use in power generation, mobility, and industry, develop international supply chains to ensure stable supply, expand renewable deployment, revitalise regional areas, and develop hydrogen related technologies \cite{noauthor_basic_2017}. The strategy aims to account for both economic and geopolitical impacts and the need to prioritise research and development to overcome the economic and technical challenges \cite{nagashima_japans_2018}. A common thread across previous research is the uncertainty surrounding \gls{CCS}, particularly with regard to scaling up and public acceptance issues, and the role that nuclear energy will play, largely due to policy reform which occurred after the Fukushima nuclear accident \cite{oshiro_mid-century_2019}.

The model and the approach proposed in this research uniquely build on the modelling consensus outlined in the literature review and expand the consideration of technologies by including emerging near-term alternatives, and potentially disruptive technologies post-2050. This work leverages the dynamic simulation capabilities of \gls{TIMES} \cite{loulou_etsap-tiam:_2008} by incorporating learning curves for parameters such as investment and \gls{OM} costs, efficiency, and emission coefficients. Our model incorporates not only direct emissions, but also lifecycle emissions of all conventional and emerging technologies, the latter of which is an often neglected externality. This enables a more meaningful analysis of the global warming potential of emerging technologies, as life-cycle emissions become significant when considering deep emission reductions of the order of 80\% from current levels. By modelling the Japanese electricity supply system out to the year 2100, our aim is to detail the mid- to long-term impacts of technological development and market penetration, and to identify the suite of technologies which could underpin the successful achievement of carbon reduction. \added{Our sensitivity analysis is another novel aspect of our research, in that it elicits important technologies and their key parameters while remaining grounded in the context of the carbon constraints and the initial conditions that are reflective of many developed nations that aim to reduce their emissions.}
