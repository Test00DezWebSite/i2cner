\section{Conclusion} \label{Conclusion}
%what we did. sum up key technologies, what needs to happen for technologies that are not "key", importance of rapid change/suggested timeline.

We simulated five transition scenarios which assess potential pathways to meeting 2030 and 2050 emission targets within the Japanese electricity supply system and their long-term impact up to 2100. \deleted{Scenario 1 (Figure \ref{scen1})} \added{Our transition scenarios} prove\deleted{d} that meeting emission goals without new nuclear or new low-emission technologies is infeasible, and such an endeavour is likely to be an expensive failure. \deleted{The remaining scenarios} \added{These results also} demonstrate that emission goals can be met by either investing heavily in nuclear \deleted{(Figure \ref{scen2})}, investing heavily in hydrogen \deleted{(Figs. \ref{scen3} and \ref{scen5})}, or using a combination of both \deleted{(Figure \ref{scen4})}. Scenarios that incorporate nuclear are the most cost-effective, and using a combination of nuclear and hydrogen leads to the greatest emission reduction post-2050. Key technologies that emerge from are results include nuclear power and hydrogen from renewables, while \gls{CCS} with natural gas and photochemical water splitting (\gls{PWS}) play a nominal role. CCS with coal, steam reforming with or without CCS, new coal, and new oil are not utilised due to their high direct and life-cycle emissions. Our analysis indicates that while politically challenging, a hybrid nuclear-hydrogen strategy is economically feasible and results in long-term emission reduction. Such a multifaceted approach to emission reduction is also likely to improve decarbonisation outcomes since the commercialisation and deployment of hydrogen in time to meet 2030 and 2050 emission goals is uncertain.

Mitigating emissions from the industrial and transportation sector presents unique challenges that may affect the amount of emission reduction required from the electricity supply sector to meet Japan's 2030 and 2050 goals. Future work should incorporate holistic assessment of the entire Japanese energy system when exploring energy transition pathways. The assessment of synergistic utilisation of hydrogen in transportation and industry alongside electricity storage and supply is vital for policy decisions. The effect of transportation media such as trucks and pipelines on hydrogen and CCS is also worth investigating. \added{Any new technologies that develop in the future and promise rapid decarbonisation should also be incorporated in such work.} Finally, economic feasibility analyses with respect to national budget requirements and projected GDP trends must also be conducted to improve decarbonisation strategies, improve social outcomes, and delineate investment goals for the energy sector.

%\section{Future work}
%grid resilience - analysis of grid stability of similar mixes at the resolution of minutes. Maybe in TIMES (ugh).
%\section{Declaration of Competing Interest}
%The authors declare no conflict of interest.