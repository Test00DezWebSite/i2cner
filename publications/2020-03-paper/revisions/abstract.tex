\begin{abstract}

We simulated possible pathways to meeting 2030 and 2050 emission targets within the Japanese electricity supply sector using a single-region model in \gls{TIMES}. Critically, our simulations incorporate novel technologies like hydrogen electrolysers, carbon capture, photochemical water splitting, and emerging photovoltaic cells, assess long-term impacts up to the year 2100, and include life-cycle emissions and learning curves for parameters such as investment cost, efficiency, and emission coefficients. Results indicate that a hybrid approach, using nuclear power and hydrogen from renewable energy-based electrolysis, is cost-effective and provides long-term emission reduction along with energy security. Nuclear, wind, solar, and hydrogen from renewables emerge as key emission reduction technologies, while natural gas with carbon capture plays a minor role in achieving emission reduction targets.

\end{abstract}
